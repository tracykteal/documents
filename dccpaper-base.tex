%%
%% This is file `dccpaper-base.tex',
%% generated with the docstrip utility.
%%
%% The original source files were:
%%
%% dccpaper.dtx  (with options: `base')
%% 
%% ----------------------------------------------------------------
%% The dccpaper bundle: Classes for submissions to IJDC and IDCC
%% Author:  Alex Ball
%% E-mail:  a.ball@ukoln.ac.uk
%% License: Released under the LaTeX Project Public License v1.3c or later
%% See:     http://www.latex-project.org/lppl.txt
%% ----------------------------------------------------------------
%% 
\def\Version{2014/08/07 v1.3}
\ProvidesFile{dccpaper-base.tex}
    [\Version\space Common class code for IJDC and IDCC papers.]
%
% The \textsf{dccpaper} classes are deliberately very similar. This file
% contains the common code. All the classes are based on the \textsf{article}
% class, and use A4 paper.
%
\LoadClass[a4paper,12pt,twoside]{article}

%
% We use British English orthography.
%
\RequirePackage[british]{babel}
%
% The macro patching commands from \textsf{etoolbox} come in useful for handling
% author and date information, and also for compatibility with \textsf{apacite}.
%
\RequirePackage{etoolbox}
%
% The \textsf{dccpaper} classes use Times as the main text font. We prefer
% \textsf{newtx} as it provides support for mathematics, but the standard
% \textsf{times} package will do. In case they are needed, we also provide
% Helvetica for the sans serif font and Computer Modern Teletype for the
% monospaced.
%
\RequirePackage[T1]{fontenc}
\IfFileExists{newtxtext.sty}%
  {\RequirePackage{newtxtext,newtxmath}}%
  {\RequirePackage{times}}
\RequirePackage[scaled=0.92]{helvet}
\renewcommand{\ttdefault}{cmtt}
%
% We will need support for included graphics and colour. The structural elements
% are a medium turquoise, while the links are slightly darker to make them
% easier to read on screen.
%
\RequirePackage{graphicx}
\IfFileExists{xcolor.sty}%
  {\RequirePackage{xcolor}}%
  {\RequirePackage{color}}
\definecolor[named]{struct}{rgb}{0,0.5,0.5}
\definecolor[named]{links}{rgb}{0,0.4,0.4}
%
% We will calculate some lengths later.
%
\RequirePackage{calc}

%
% Ragged right text is easier to read on screen, but fully justified text looks
% better. The |\raggedyright| layout from Peter Wilson's \textsf{memoir} class
% (2005/09/25 v1.618) is an excellent compromise. The code below replicates it.
%
% First we save the original definitions of |\\| and |\parindent| as
% |\OrigLineBreak| and |\RaggedParindent| respectively.
%
\let\OrigLineBreak\\
\newdimen\RaggedParindent
\setlength{\RaggedParindent}{\parindent}

%
% The |\raggedyright| layout more or less lays text out as with full
% justification, but then lets the shorter lines relax a bit from the right
% margin. It is the default for DCC papers.
%
\newcommand{\raggedyright}[1][2em]{%
  \let\\\@centercr\@rightskip \z@ \@plus #1\relax
  \rightskip\@rightskip
  \leftskip\z@skip
  \parindent\RaggedParindent}
\AtBeginDocument{\raggedyright}

%
% The |\flushleftright| layout restores full justification, in case it is
% needed.
%
\newcommand*{\flushleftright}{%
  \let\\\OrigLineBreak
  \leftskip\z@skip
  \rightskip\leftskip
  \parfillskip\@flushglue
  \everypar{}}

%
% The classes have some special metadata requirements. We start with the author
% information.
%
% The macro |\thecorrespondence| is used in the title page footer for the name,
% postal address and email address of the corresponding author.
%
\def\thecorrespondence{}
\newcommand*{\correspondence}[1]{\def\thecorrespondence{#1}}
%
% The handling of authors here is inspired by Patrick W Daly's \textsf{authblk},
% (2001/02/27 1.3), and defines the familiar user commands. Authors are
% presented in blocks, one affiliation but perhaps several authors per block.
%
% We make the presentation of the author information configurable (just in case)
% with some hooks and lengths:
% \begin{itemize}
% \item |\Authfont| is the font used for author names;
% \item |\Affilfont| is the font used for affiliations;
% \item |\affilsep| is the line spacing between author names and affiliations;
% \item |\authblksep| is the line spacing between author name/affiliation
%       blocks.
% \end{itemize}
%
\newcommand*{\Authfont}{\normalfont}
\newcommand*{\Affilfont}{\normalfont\small}
\newlength{\affilsep}\setlength{\affilsep}{0pt}
\newlength{\authblksep}\setlength{\authblksep}{1.5\baselineskip}
%
% The |\AuthorBlock| command will be used to collect all the author information.
%
\newcommand{\AuthorBlock}{}
%
% The |\MainAuthor| command collects abbreviated author information for use in
% the headers.
%
\newcommand{\MainAuthor}{}
%
% The |\dccp@author| command, meanwhile, collects a full list of authors for the
% PDF metatdata.
%
\newcommand{\dccp@author}{}
%
% We define counters for
% \begin{itemize}
% \item the total number of authors defined;
% \item the number of authors in the current block;
% \item the number of blocks defined.
% \end{itemize}
%
\newcounter{authors}
\newcounter{authorsinblock}
\newcounter{block}
%
% The |block| counter will increase once in the preamble, and again when the
% information is typeset, so we need to reset it at the beginning of the
% document.
%
\AtBeginDocument{\setcounter{block}{0}}
%
% The new definition of the author command starts here.
%
\renewcommand{\author}[1]{%
%
% If this is the first or second |\author| command, we add the name to our
% abbreviated list of authors. Otherwise, we replace the name of the second
% and subsequent authors with `et al.' in that list.
%
% At the same time, we use a simpler technique to populate |\dccp@author|.
%
  \ifnum\theauthors=0
    \def\dccp@author{#1}%
    \def\MainAuthor{#1}%
  \else
    \appto\dccp@author{, #1}%
    \ifnum\theauthors=1%
      \def\OtherMainAuthors{ and #1}%
      \appto\MainAuthor{\OtherMainAuthors}%
    \else
      \ifnum\theauthors=2%
        \def\OtherMainAuthors{ et al.}%
      \fi
    \fi
  \fi
  \stepcounter{authors}%
%
% Each block has its author names collected in a macro like |\blocki@auth|,
% and its affiliation collected in a macro like |\blocki@affil| (the `i' is a
% serial number).
%
% If this is the first author in a block, we need to create the block and add it
% to |\AuthorBlock|; second and subsequent blocks are preceded by a |\quad| of
% space.
%
  \ifnum\theauthorsinblock=0%
    \stepcounter{block}%
    \expandafter\def\csname block\roman{block}@auth\endcsname{#1}%
    \ifnum\theblock>1\appto\AuthorBlock{\quad}\fi
    \appto\AuthorBlock{%
      \stepcounter{block}%
      \begin{minipage}[t]{0.45\textwidth}\centering
      \csname block\roman{block}@auth\endcsname
      \ifx\undefined\csname block\roman{block}@affil\endcsname
      \else
        \\[\affilsep]\csname block\roman{block}@affil\endcsname
      \fi
      \end{minipage}%
      \rule{0pt}{2\baselineskip}%
      }
  \else
%
% Otherwise we just add the name to the right |\blocki@auth|-style macro.
%
    \csappto{block\roman{block}@auth}{, #1}%
  \fi
  \stepcounter{authorsinblock}%
}
%
% The |\affil| command adds an affiliation to the current block and closes it by
% resetting the |authorsinblock| counter.
%
\newcommand{\affil}[1]{%
  \expandafter\def\csname block\roman{block}@affil\endcsname{\Affilfont#1}%
  \setcounter{authorsinblock}{0}%
}

%
% The |\HeadTitle| collects the abbreviated title for use in the headers.
%
\newcommand*{\HeadTitle}{}
%
% We wrap the normal |\title| command with code to populate |\HeadTitle| with
% the optional argument if provided, or the mandatory one otherwise. We also
% provide a persistent |\thetitle| macro, stripped of any |\thanks|.
%
\let\ProperTitle=\title
\renewcommand{\title}[2][\empty]{
  \ifx\empty #1%
    \renewcommand*{\HeadTitle}{#2}%
  \else
    \renewcommand*{\HeadTitle}{#1}%
  \fi%
  \begingroup\let\footnote\@gobble
  \ProperTitle{#2}%
  \begingroup
    \renewcommand{\thanks}[1]{}
    \protected@xdef\thetitle{#2}
  \endgroup\endgroup
}

%
% We make |\thedate| persistent, borrowing the technique used in Peter Wilson's
% \textsf{memoir} class (2005/09/25 v1.618).
%
\pretocmd{\date}{\begingroup\let\footnote\@gobble}{}{}%
\apptocmd{\date}{%
  \begingroup
    \renewcommand{\thanks}[1]{}
    \protected@xdef\thedate{#1}
  \endgroup\endgroup%
}{}{}
%
% We ensure |\thedate| is defined using a dummy date.
%
\date{20xx}

%
% IJDC articles have extra bibliographic information:
% \begin{itemize}
% \item |\volume| sets the volume number, |\thevolume|;
% \item |\issue| sets the issue number, |\theissue|;
% \item |\subno| sets the submission number, |\thesubno|.
% \end{itemize}
%
% These numbers are used to build the DOI, |\thedoi|.
%
\newcommand*{\thevolume}{0}
\newcommand*{\volume}[1]{\renewcommand*{\thevolume}{#1}}
\newcommand*{\theissue}{0}
\newcommand*{\issue}[1]{\renewcommand*{\theissue}{#1}}
\newcommand*{\thesubno}{0}
\newcommand*{\subno}[1]{\renewcommand*{\thesubno}{#1}}
\newcommand*{\thedoi}{10.2218/ijdc.v\thevolume i\theissue .\thesubno}
%
% They also display the page range. The following code was borrowed from Peter
% Wilson's \textsf{memoir} class (2005/09/25 v1.618). It defines a counter
% |lastpage| which, on the second run, will contain the number of the last page.
%
\newcounter{lastpage}
\setcounter{lastpage}{0}
\newcommand{\dol@stpage}{%
  \if@filesw
    \addtocounter{page}{-1}%
    \immediate\write\@auxout%
      {\string\setcounter{lastpage}{\the\c@page}}%
    \stepcounter{page}%
  \fi
}
\AtBeginDocument{\AtEndDocument{\clearpage\dol@stpage}}

%
% DCC papers display some important dates. We collect these in |\dccp@dates|,
% initially setting the value to something sensible for papers in draft.
%
\def\dccp@dates{\emph{Draft from} \today}
%
% Several types of date can be added:
% \begin{itemize}
% \item |\submitted| for when the authors submitted the paper (intended for
%       IDCC papers).
% \item |\received| for when the paper was received by the editorial board
%       (intended for IJDC papers).
% \item |\revised| for when the most recent version was received by the
%       editorial board.
% \item |\accepted| for when the paper was accepted by the editorial board.
% \end{itemize}
%
\newcommand*{\submitted}[1]{%
  \def\dccp@dates{\emph{Submitted} #1}}
\newcommand*{\received}[1]{%
  \def\dccp@dates{\emph{Received} #1}}
\newcommand*{\revised}[1]{%
  \appto\dccp@dates{%
    \space\space\space\textbar\space\space\space
    \emph{Revision received} #1}%
  }
\newcommand*{\accepted}[1]{%
  \appto\dccp@dates{%
    \space\space\space\textbar\space\space\space
    \emph{Accepted} #1}%
  }
%
% IJDC papers need to say if they had a previous life as a conference paper.
% This statement goes in |\dccp@conf|, which is initially empty. The user
% command for setting this text is |\conference|.
%
\let\dccp@conf=\empty
\newcommand*{\conference}[1]{%
  \renewcommand*{\dccp@conf}{An earlier version of this paper was presented at #1.}%
}

%
% The macro |\FixTextHeight| will be useful when switching from the first page
% geometry to the regular geometry for the rest of the paper. It is based on
% code from  Hideo Umeki's \textsf{geometry} package (2002/07/08 v3.2).
%
\newcommand{\FixTextHeight}{%
  \setlength\@tempdima{\textheight}%
  \addtolength\@tempdima{-\topskip}%
  \@tempcnta\@tempdima
  \@tempcntb\baselineskip
  \divide\@tempcnta\@tempcntb
  \setlength\@tempdimb{\baselineskip}%
  \multiply\@tempdimb\@tempcnta
  \advance\@tempdima-\@tempdimb
  \global\advance\footskip\@tempdima
  \multiply\@tempdima\tw@
  \ifdim\@tempdima>\baselineskip
    \addtolength\@tempdimb{\baselineskip}%
    \global\advance\footskip-\baselineskip
  \fi
  \addtolength\@tempdimb{\topskip}%
  \global\textheight\@tempdimb
}

%
% The width of the textblock (on all pages) is 150mm, which on A4 paper implies
% margins of 30mm each. (Making both horizontal margins the same in a two-sided
% context makes the paper more pleasant to read on screen).
%
\setlength{\textwidth}{150mm}
\setlength{\oddsidemargin}{30mm - \hoffset - 1in}
\setlength{\evensidemargin}{30mm - \hoffset - 1in}
%
% It is rare to have marginal notes, but in case we ever do, we centre them in
% the margin.
%
\setlength{\marginparwidth}{30mm - 2\marginparsep}
%
% We also want a distance of 15mm from the top of the page to the top of the
% header, and two blank lines between the bottom of the header and the top of
% the textblock.
%
\setlength{\topmargin}{15mm - \voffset - 1in}
\setlength{\headsep}{2\baselineskip}

%
% IJDC editorials have slightly different headers and footers. This requires
% testing for |\dccp@editorial| if it exists. In case it doesn't, we provide it.
%
\providecommand{\dccp@editorial}{Editorial}
%
% The height of the footer can vary a lot. To keep it a fixed distance from the
% bottom of the page rather than the top, we need to vary the |\textheight|
% accordingly. This means we need to measure the height of the footer. (The
% header is more predictable but we may as well measure it while we are at it).
%
% Here we define the header and footer of the title page (i.e.\ the
% \textsf{title} page style), making sure we save them to auxiliary macros
% |\TitleHead| and |\TitleFoot| so we can measure them.
%
\def\ps@title{%
  \def\@oddhead{%
    \begin{minipage}{\textwidth}%
      \centering
      \LARGE\bfseries\color{struct}%
      \ifx\dccp@type\dccp@editorial
        \dccp@publ@long
      \else
        \dccp@publ@short\space\space\textbar\space\space\emph{\dccp@type}%
      \fi
      \par
    \end{minipage}%
  }%
  \let\@evenhead=\@oddhead
  \let\TitleHead=\@oddhead
  \def\@oddfoot{%
    \begin{minipage}[b]{\textwidth}%
      \fontsize{9pt}{11pt}\selectfont
      \ifx\dccp@type\dccp@editorial
      \else
        {\centering\dccp@dates\par}
        \bigskip
        Correspondence should be addressed to \thecorrespondence\par
        \bigskip
      \fi
      \ifx\empty\dccp@conf
      \else
        \dccp@conf\par
        \bigskip
      \fi
      \dccp@publ@msg\par
      \bigskip
      \begin{minipage}[b]{\linewidth - 25mm}
      Copyright rests with the authors. This work is released under a Creative
      Commons Attribution 4.0 International Licence. For details please see
      \url{http://creativecommons.org/licenses/by/4.0/}%
      \end{minipage}\hfill
      \begin{minipage}[b]{19mm}
        \href{http://creativecommons.org/licenses/by/4.0/}%
          {\includegraphics[width=\hsize]{dccpaper-by}}%
      \end{minipage}
      \par
      \bigskip
      \makebox[0pt][l]{\parbox{0.4\hsize}{%
        \ifx\undefined\dccp@titlefoot@bib\else\dccp@titlefoot@bib\fi
      }}\hfill
      \makebox[0pt][c]{\normalsize\thepage}\hfill
      \makebox[0pt][r]{\parbox{0.4\hsize}{%
        \raggedleft\ifx\undefined\dccp@titlefoot@doi\else\dccp@titlefoot@doi\fi
      }}%
    \end{minipage}%
  }%
  \let\@evenfoot=\@oddfoot
  \let\TitleFoot=\@oddfoot
}
%
% We set the normal page style to \textsf{title} here so that |\TitleHead| and
% |\TitleFoot| are defined, but we will override it with the \textsf{dccpaper}
% page style later.
%
\pagestyle{title}
%
% The first page should use the \textsf{title} page style, however.
%
\AtBeginDocument{\thispagestyle{title}}

%
% Here are the normal headers and footers (i.e.\ the \textsf{dccpaper} page
% style). We save them to |\NormalHead| and |\NormalFoot|, again so we can
% measure them.
%
\def\ps@dccpaper{%
  \def\@oddhead{%
    \begin{minipage}{\textwidth}\frenchspacing
      {%
        \fontsize{9pt}{11pt}\selectfont
        \ifx\undefined\dccp@normhead@doi\else\dccp@normhead@doi\fi
      }\hfill
      {\MainAuthor}\space\space\space
      \textcolor{struct}{\textbar}\space\space\space
      \thepage\par
      \vskip6pt\color{struct}{\hrule height 1bp}\par
    \end{minipage}
  }%
  \def\@evenhead{%
    \begin{minipage}{\textwidth}
      \thepage\space\space\space
      \textcolor{struct}{\textbar}\space\space\space
      {\HeadTitle}\hfill
      {%
        \fontsize{9pt}{11pt}\selectfont
        \ifx\undefined\dccp@normhead@doi\else\dccp@normhead@doi\fi
      }\par
      \vskip6pt\color{struct}{\hrule height 1bp}\par
    \end{minipage}
  }%
  \let\NormalHead=\@oddhead
  \def\@oddfoot{\begin{minipage}[b]{\textwidth}
    \centering\bfseries\normalsize\color{struct}
    \ifx\dccp@type\dccp@editorial
      \dccp@publ@long
    \else
      \dccp@publ@short\space\space\textbar\space\space\emph{\dccp@type}%
    \fi
    \par
  \end{minipage}}%
  \let\@evenfoot=\@oddfoot
  \let\NormalFoot=\@oddfoot
}
\pagestyle{dccpaper}

%
% We need to wait until the author has supplied the necessary information before
% we can do our measuring and set the remainder of the geometry, so we do it at
% the end of the preamble. First we put our saved macros into boxes we can
% measure (i.e.\ |\dccp@firstpagehead|, |\dccp@firstpagefoot|,
% |\dccp@restpagehead|, |\dccp@restpagefoot|).
%
\AtEndPreamble{
  \newsavebox{\dccp@firstpagehead}
  \sbox\dccp@firstpagehead{\normalfont\TitleHead}
  \newsavebox{\dccp@firstpagefoot}
  \sbox\dccp@firstpagefoot{\normalfont
    \def\email#1{#1}\def\url#1{#1}\def\href#1#2{#2}\TitleFoot}
  \newsavebox{\dccp@restpagehead}
  \sbox\dccp@restpagehead{\normalfont\NormalHead}
  \newsavebox{\dccp@restpagefoot}
  \sbox\dccp@restpagefoot{\normalfont\NormalFoot}
%
% We can now set the geometry of the title page\dots
%
  \setlength{\headheight}{\ht\dccp@firstpagehead + \dp\dccp@firstpagehead}
  \setlength{\footskip}{%
    2\baselineskip + \ht\dccp@firstpagefoot + \dp\dccp@firstpagefoot
  }
  \setlength{\textheight}{%
    \paperheight
    - 30mm % 15mm top and bottom
    - \headheight
    - \headsep
    - \footskip
  }
%
% {\dots}and provide a macro that will reset the geometry for the remaining
% pages.
%
  \def\dccp@resetgeometry{%
    \setlength{\headheight}{\ht\dccp@restpagehead + \dp\dccp@restpagehead}
    \global\headheight=\headheight
    \setlength{\footskip}{%
      2\baselineskip + \ht\dccp@restpagefoot
    }
    \global\footskip=\footskip
    \setlength{\textheight}{%
      \paperheight
      - 30mm % 15mm top and bottom
      - \headheight
      - \headsep
      - \footskip
    }
    \FixTextHeight
    \global\textheight=\textheight
  }
}

%
% The |\maketitle| command is redefined to the correct formatting. At the end it
% sets a hook that will reset the geometry when the first page is shipped out,
% i.e.\ with effect from the second page. It is here rather than at the end of
% the abstract in case the abstract itself spills over to the second page.
%
\RequirePackage{atbegshi}
\renewcommand{\maketitle}{%
  \null\nobreak\vspace*{-0.528\baselineskip}%
  \begingroup
    \centering
    {\Large\thetitle\par}
    \vspace{0.7\baselineskip}
    \AuthorBlock\par
    \vspace{1.7\baselineskip}
  \endgroup
  \AtBeginShipoutNext{\dccp@resetgeometry}%
}

%
% The \textsf{abstract} environment is redefined in terms of an environment
% \textsf{widequote}, which mimics the \textsf{quote} environment, but is a bit
% wider. We also provide a hook, |\afterabstract|, so that if some annotation
% needs to be appended to the title page after the abstract, we can do that.
%
\newenvironment{widequote}{%
  \list{}{%
    \setlength{\rightmargin}{2\parindent}%
    \setlength{\leftmargin}{2\parindent}%
  }%
  \flushleftright\item[]%
}{%
    \endlist
}
\def\afterabstract{}
\renewenvironment{abstract}{%
  \vskip1em%
  \begin{center}%
    {\bfseries\abstractname\vspace{-.5em}\vspace{\z@}}%
  \end{center}%
  \widequote\footnotesize
}{%
  \endwidequote\afterabstract\newpage
}

%
% We use the \textsf{titlesec} package to give headings the correct formatting.
% The settings below try to space out headings so they occupy an integer number
% of normal lines (an attempt at grid typesetting). They are a little
% complicated because we want it to work even if the heading appears at the top
% of the page.
%
\RequirePackage{titlesec}
\titlespacing*{\section}{0pt}{0pt}{\baselineskip}
\titlespacing*{\subsection}{0pt}{0pt}{0.6\baselineskip}
\titlespacing{\subsubsection}{\parindent}{\baselineskip}{0pt}
\titlespacing{\paragraph}{\parindent}{\baselineskip}{0pt}
\titlespacing{\subparagraph}{\parindent}{\baselineskip}{0pt}
%
% An unfortunate side effect of spacing headings like this is that if a
% |\subsection| immediately follows a |\section| it forms an unsightly gap. To
% remedy this, we count how many paragraphs there have been since the last
% |\section|. Note that as we do not normally number the sections, an automatic
% reset of the |sectionpars| counter within the |section| counter won't work.
%
\newcounter{sectionpars}
\let\dccp@old@ep\everypar
\newtoks\everypar
\dccp@old@ep{\the\everypar\stepcounter{sectionpars}}
%
% We need to manually reset |sectionpars| when |\section| is called. Also,
% the normal font size is 12pt/14.5pt, while |\Large| is 17pt/22pt;
% so the |\Large| line height = 1.5172 $\times$ normal line height. Nevertheless
% it seems to work better if we let the heading eat 0.528|\baselineskip| into
% the 2|\baselineskip| of padding above it.
%
\titleformat{\section}
  [block]
  {%
    \vspace{2\baselineskip}%
    \nobreak
    \vspace*{-0.528\baselineskip}%
    \setcounter{sectionpars}{0}%
    \filcenter\normalfont\Large\bfseries
  }
  {\thesection}
  {\quad}
  {}
%
% The others use a |\normalsize| font so that makes life easier. The format
% for |\subsection| command includes conditional spacing: if the |sectionpars|
% counter equals 2, this means the heading immediately follows a |\section|, so
% less white space is needed.
%
\titleformat{\subsection}
  {%
    \ifnum\thesectionpars>2%
      \vspace{2\baselineskip}%
    \else
      \vspace{\baselineskip}%
    \fi\nobreak
    \vspace*{-0.6\baselineskip}%
    \normalfont\normalsize\bfseries
  }
  {\thesubsection}
  {\quad}
  {}
\titleformat{\subsubsection}
  [block]
  {\normalfont\normalsize\bfseries}
  {\thesubsubsection}
  {\quad}
  {}
\titleformat{\paragraph}
  [block]
  {\normalfont\normalsize\bfseries\itshape}
  {\thesubsubsection}
  {\quad}
  {}
\titleformat{\subparagraph}
  [block]
  {\normalfont\normalsize\itshape}
  {\thesubsubsection}
  {\quad}
  {}
%
% DCC papers do not typically number their sections.
%
\setcounter{secnumdepth}{0}

%
% To help with the display of tables we load the \textsf{array} and
% \textsf{booktabs} packages. As we don't like lines between rows in the table
% body, we stretch them out a bit so that white space does the job instead.
%
\RequirePackage{array,booktabs}
\renewcommand{\arraystretch}{1.25}

%
% We use the \textsf{caption} package to give captions the right format.
%
\RequirePackage
  [ format=hang
  , labelsep=period
  , font=small
  , labelfont=bf
  , figureposition=bottom
  , tableposition=top
  ]{caption}

%
% Footnotes should be set right up against the left margin. They should be
% set hung and in the same half-ragged style as the main text. They should also,
% for neatness, be at the bottom of the page regardless of how short it is. The
% \textsf{footmisc} package helps here.
%
\RequirePackage[hang,bottom]{footmisc}
\settowidth{\footnotemargin}{\footnotesize\textsuperscript{99}\space}
\renewcommand{\footnotelayout}{\raggedyright}
%
% Also, if multiple footnotes are set at once, the markers should be separated
% with superscript commas. The \textsf{footmisc} package should help here but
% its solution is clobbered by \textsf{hyperref}. So after a footnote is set,
% we check to see if the next token is also a footnote, and if so, slip a comma
% in before it.\footnote{This solution was provided at
% \url{http://tex.stackexchange.com/q/40072}} This tweak needs to be done late,
% |\AtBeginDocument|. Note that the \textsf{newtx} superior figures are a bit
% lower than normal superscript text.
%
\AtBeginDocument{
  \let\dccp@footnote\footnote
  \def\dccp@next@token{\relax}%
  \def\dccp@supercomma{\textsuperscript{,}}%
  \IfFileExists{newtxtext.sty}%
    {\def\dccp@supercomma{\raisebox{-0.2ex}{\textsuperscript{,}}}}%
    {}

  \newcommand\dccp@check@for@footnote{%
    \ifx\footnote\dccp@next@token
      \dccp@supercomma
    \fi
  }

  \renewcommand\footnote[1]{%
    \dccp@footnote{#1}%
    \futurelet\dccp@next@token\dccp@check@for@footnote
  }
}

%
% By default lists are quite loose. These settings help to tighten them.
%
\topsep = \z@
\partopsep = \z@
\appto{\enumerate}{\itemsep = 0.5ex plus 0.25ex minus 0.25ex}
\appto{\itemize}{\itemsep = 0.5ex plus 0.25ex minus 0.25ex}

%
% A DCC paper should either be using \textsf{biblatex} or \textsf{apacite} for
% references.
%
% If \textsf{biblatex} is used, we need to ensure that the reference list
% heading is a normal section rather than a starred one so it appears in the
% PDF bookmarks.
%
\AtBeginDocument{
  \@ifpackageloaded{biblatex}{%
    \defbibheading{bibliography}[\refname]{\section{#1}}%
  }{%
%
% If \textsf{apacite} is used, we can do the same with a package option (see
% below). But there are a few other adaptations we need to make.
%
    \@ifpackageloaded{apacite}{%
%
% With \textsf{hyperref} loaded, \textsf{apacite} makes the whole of a citation
% a link to the reference list item. We patch |\@ifauthorsunequalc@de| so only
% the year portion gets linked.
%
      \def\@ifauthorsunequalc@de#1{%
        \if@F@cite
          \@F@citefalse
        \else
          \if@Y@cite
              {\@BAY}%
          \fi
          {\@BBC}%
        \fi
        \edef\@cite@undefined{?}%
        \def\BBA{\@BBA}%
        \if@A@cite
          %%\hyper@natlinkstart{#1}% We remove this line...
          {\csname b@\@citeb\APAC@extra@b@citeb\endcsname}%
          %%\hyper@natlinkend% ...and this one.
          \if@Y@cite
              {\@BBAY}%
          \fi
        \fi
        \if@Y@cite
          \hyper@natlinkstart{#1}%
          {\csname Y@\@citeb\APAC@extra@b@citeb\endcsname}%
          \hyper@natlinkend
        \fi
        \let\BBA\relax
      }
%
% The Spanish language support file defines a different version of
% |\@ifauthorsunequalc@de|, which might override the patch we have just
% introduced. So we employ the same test that \textsf{apacite} uses when
% deciding whether to load that file; if successful, we patch the Spanish
% version. Note that as \textsf{apacite} loads language support files
% |\AtBeginDocument|, we have to do our thing after that, |\AfterEndPreamble|.
%
% (Note that as we set the language to British English earlier, this should
% never be needed, but we try to be resilient to tinkering!)
%
      \AfterEndPreamble{%
        \@ifundefined{iflanguage}{%
          \relax
        }{%
          \edef\APAC@tmp{nohyphenation}%
          \ifx\languagename\APAC@tmp
          \else
            \edef\APAC@tmp{spanish}%
            \ifx\languagename\APAC@tmp
              \def\@ifauthorsunequalc@de#1{%
                \if@F@cite
                  \@F@citefalse
                \else
                  \if@Y@cite
                      {\@BAY}%
                  \fi
                  {\@BBC}%
                \fi
                \edef\@cite@undefined{?}%
                \def\BBA{\@BBA}%
                \@ifundefined{spanishe@\@citeb\APAC@extra@b@citeb}%
                  {}% skip
                  {{% Use `e' instead of `y' in Spanish
                  \global\let\oldBBA\BBA
                  \global\def\BBA{e\global\let\BBA\oldBBA}%
                  }}%
                \if@A@cite
                  %%\hyper@natlinkstart{#1}% We remove this line...
                  {\csname b@\@citeb\APAC@extra@b@citeb\endcsname}%
                  %%\hyper@natlinkend% ...and this one.
                  \if@Y@cite
                      {\@BBAY}%
                  \fi
                \fi
                \if@Y@cite
                  \hyper@natlinkstart{#1}%
                  {\csname Y@\@citeb\APAC@extra@b@citeb\endcsname}%
                  \hyper@natlinkend
                \fi
                \let\BBA\relax
              }%
            \fi
          \fi
        }%
%
% Another thing \textsf{apacite} does |\AtBeginDocument| is set the URL style
% to monospaced. So we reset it back to normal roman type |\AfterEndPreamble|.
%
        \urlstyle{APACrm}
       }%
%
% We pre-empt \textsf{apacite}'s |\providecommand| of |\doi| with our own
% definition that includes the `doi' URI scheme label in the link, remembering
% to remove the one inserted by |\doiprefix|.
%
      \newcommand{\doi}[1]{\href{http://dx.doi.org/#1}{\nolinkurl{doi:#1}}}%
      \renewcommand{\doiprefix}{\unskip}%
    }{}%
  }%
%
% Both \textsf{biblatex} and \textsf{apacite} use |\bibitemsep| for the space
% between bibliography items. Just in case they haven't been loaded, though, we
% protect our setting of that length with an |\ifx| test.
%
  \ifx\undefined\bibitemsep
  \else
    \setlength{\bibitemsep}{1em plus 1ex minus 1ex}%
  \fi
}
%
% As mentioned above, if \textsf{apacite} is used, we can use a package option
% to ensure that the reference list heading appears in the PDF bookmarks.
%
\PassOptionsToPackage{numberedbib}{apacite}

%
% We, of course, use \textsf{hyperref} for enhancing the PDF with working links,
% bookmarks, metadata, etc.
%
\usepackage
  [ colorlinks=true
  , linkcolor=black
  , anchorcolor=black
  , citecolor=links
  , filecolor=black
  , menucolor=black
  , runcolor=black
  , urlcolor=links
  ]{hyperref}
%
% Links should be in roman type, not monospaced.
%
\urlstyle{rm}
%
% We provide an |\email| command for displaying the email address of the
% corresponding author.
%
\newcommand*{\email}[1]{\href{mailto:#1}{#1}}
%
% Once the user has had a chance to provide the metadata, we can add it to the
% PDF metadata.
%
\AtBeginDocument{%
  \hypersetup
    { pdftitle={\thetitle}
    , pdfauthor={\dccp@author}
    , pdfsubject={\dccp@subject}
    }
%
% The APA has its own style for line breaks in URLs. The \textsf{apacite}
% package provides the code for this, but in case \textsf{biblatex} is used
% instead, we repeat the settings (from 2013/07/21 v6.03) here.
%
  \@ifundefined{Url@force@Tilde}{\def\Url@force@Tilde{\relax}}{}%
  \def\url@apa@dot{\mathchar"2E }%
  \def\url@apa@comma{\mathchar"2C }%
  \def\url@apa@questionmark{\mathchar"3F }%
  \def\url@apa@exclamation{\mathchar"21 }%
  \def\url@apa@hyphen{\mathchar"2D }%
  \def\url@apa@underscore{\_}%
  \def\UrlBreaks{\do\@\do\\\do\|\do\;\do\>\do\]\do\)\do\'\do+\do\=\do\#}%
  \def\UrlBigBreaks{\do\/\do\:\do@url@hyp}%
  \def\UrlNoBreaks{\do\(\do\[\do\{\do\<}% \)}
  \def\UrlOrds{\do\*\do\~\do\'\do\"}%
  \def\UrlSpecials{%
    \do\.{\mathbin{}\url@apa@dot }%
    \do\,{\mathbin{}\url@apa@comma }%
    \do\-{\mathbin{}\url@apa@hyphen }%
    \do\?{\mathbin{}\url@apa@questionmark }%
    \do\!{\mathbin{}\url@apa@exclamation }%
    \do\_{\mathbin{}\url@apa@underscore }%
    \do\ {\Url@space}\do\%{\Url@percent}\do\^^M{\Url@space}%
    \Url@force@Tilde}%
  \def\Url@OTnonTT{\do\<{\langle}\do\>{\mathbin{\rangle}}\do
    \_{\mathbin{}\_}\do\|{\mid}\do\{{\lbrace}\do\}{\mathbin{\rbrace}}\do
    \\{\mathbin{\backslash}}\UrlTildeSpecial}
}

%
% We now embed the Creative Commons licence information in the PDF using an XMP
% packet. To do this, we employ the same technique as Scott Pakin's
% \textsf{hyperxmp} (2014/01/02 v2.4). In order to avoid avoid a bug whereby
% Adobe Acrobat confuses the XMP author information and the regular author
% information, though, we \emph{only} embed the licence information.
%
% We need to make sure that any characters to appear verbatim in the XMP packet
% are treated as ordinary characters and not active ones. The likely active
% characters are symbols and punctuation, so should be treated as `other'
% (category 12).
%
\begingroup
\catcode`\"=12
\catcode`\&=12
\catcode`\#=12
\catcode`\<=12
\catcode`\>=12
\catcode`\_=12
%
% We construct the XMP packet as the document begins.
%
\AtBeginDocument{%
%
% For convenience we define |\sp| to be a level of indent, translating to three
% spaces.
%
  \def\sp{\space\space\space}
%
% The text of the XMP packet is recorded in |\cc@xmp@packet|. We use |^^J| to
% break lines.
%
  \long\gdef\cc@xmp@packet{%
<?xpacket begin='' id=''?>^^J%
<x:xmpmeta xmlns:x='adobe:ns:meta/'>^^J%
<rdf:RDF xmlns:rdf='http://www.w3.org/1999/02/22-rdf-syntax-ns#'>^^J%
\sp<rdf:Description rdf:about=''^^J%
\sp\sp xmlns:xapRights='http://ns.adobe.com/xap/1.0/rights/'>^^J%
\sp\sp<xapRights:Marked>True</xapRights:Marked>^^J%
\sp</rdf:Description>^^J%
\sp<rdf:Description rdf:about=''^^J%
\sp\sp xmlns:dc='http://purl.org/dc/elements/1.1/'>^^J%
\sp\sp<dc:rights>^^J%
\sp\sp\sp<rdf:Alt>^^J%
\sp\sp\sp\sp<rdf:li xml:lang='x-default'>This work is licensed under a Creative Commons Attribution 4.0 International Licence.</rdf:li>^^J%
\sp\sp\sp</rdf:Alt>^^J%
\sp\sp</dc:rights>^^J%
\sp</rdf:Description>^^J%
\sp<rdf:Description rdf:about=''^^J%
\sp\sp xmlns:cc='http://creativecommons.org/ns#'>^^J%
\sp\sp<cc:license rdf:resource='http://creativecommons.org/licenses/by/4.0/'/>^^J%
\sp</rdf:Description>^^J%
</rdf:RDF>^^J%
</x:xmpmeta>^^J%
<?xpacket end='r'?>^^J%
  }%
}
\endgroup
%
% Different workflows require the XMP packet to be embedded in different ways.
%
% Pdf\TeX\ can inject objects into PDFs natively.
%
\newcommand*{\ccxmp@embed@packet@pdftex}{%
  \bgroup
    \pdfcompresslevel=0
    \immediate\pdfobj stream attr {%
      /Type /Metadata
      /Subtype /XML
    }{\cc@xmp@packet}%
    \pdfcatalog {/Metadata \the\pdflastobj\space 0 R}%
  \egroup
}
%
% The |\pdfmark| command defined by \textsf{hyperref} is respected by tools such
% as Dvipdf, Dvips, Dvipsone, etc.
%
\newcommand*{\ccxmp@embed@packet@pdfmark}{%
  \pdfmark{%
    pdfmark=/NamespacePush
  }%
  \pdfmark{%
    pdfmark=/OBJ,
    Raw={/_objdef \string{ccxmp@packet\string} /type /stream}%
  }%
  \pdfmark{%
    pdfmark=/PUT,
    Raw={\string{ccxmp@packet\string}
      2 dict begin
        /Type /Metadata def
        /Subtype /XML def
        currentdict
      end
    }%
  }%
  \pdfmark{%
    pdfmark=/PUT,
    Raw={\string{ccxmp@packet\string} (\cc@xmp@packet)}%
  }%
  \pdfmark{%
    pdfmark=/Metadata,
    Raw={\string{Catalog\string} \string{ccxmp@packet\string}}%
  }%
  \pdfmark{%
    pdfmark=/NamespacePop
  }%
}
%
% Dvipdfm has its own |\special| command for inserting PDF objects, but
% it is a bit basic and requires advance knowledge of how long (in characters)
% the object is.
%
% The |\ccxmp@count@spaces| macro counts the number of spaces in its parameter
% through a process of iteration, adding this figure to |\@tempcnta|.
%
\def\ccxmp@count@spaces#1 {%
  \def\ccxmp@one@token{#1}%
  \ifx\ccxmp@one@token\@empty
    \advance\@tempcnta by -1
  \else
    \advance\@tempcnta by 1
    \expandafter\ccxmp@count@spaces
  \fi
}
%
% The |\ccxmp@count@non@spaces| command counts the number of non-spaces in its
% argument through a process of iteration, adding this figure to |\@tempcnta|.
%
\newcommand*{\ccxmp@count@non@spaces}[1]{%
  \def\ccxmp@one@token{#1}%
  \ifx\ccxmp@one@token\@empty
  \else
    \advance\@tempcnta by 1
    \expandafter\ccxmp@count@non@spaces
  \fi
}
%
% The |\ccxmp@string@len| command sets |\@tempcnta| to the number of characters
% (spaces + non-spaces) in its argument.
%
\newcommand*{\ccxmp@string@len}[1]{%
  \@tempcnta=0
  \expandafter\ccxmp@count@spaces#1 {} %
  \expandafter\ccxmp@count@non@spaces#1{}%
}
%
% So now, finally, is the command for embedding the packet using Dvipdfm.
%
\newcommand*{\ccxmp@embed@packet@dvipdfm}{%
  \ccxmp@string@len{\cc@xmp@packet}%
  \special{pdf: object @ccxmp@packet
    <<
      /Type /Metadata
      /Subtype /XML
      /Length \the\@tempcnta
    >>
    stream^^J\cc@xmp@packet endstream%
  }%
  \special{pdf: docview
    <<
      /Metadata @ccxmp@packet
    >>
  }%
}
%
% \XeTeX\ creates PDFs with Xdvipdfmx, which supports a simpler |\special| for
% inserting objects that does not require us to count characters.
%
\newcommand*{\ccxmp@embed@packet@xetex}{%
  \special{pdf:stream @ccxmp@packet (\cc@xmp@packet)
    <<
      /Type /Metadata
      /Subtype /XML
    >>
  }%
  \special{pdf:put @catalog
    <<
      /Metadata @ccxmp@packet
    >>
  }%
}
%
% We rely on \textsf{hyperref} to tell us how the PDF will be generated (after
% all, it may not be done in the current pass) and use the respective technique
% to embed the XMP packet.
%
\AtBeginDocument{%
  \begingroup
  \def\ccxmp@driver{hpdftex}%
  \ifx\ccxmp@driver\Hy@driver
    \ccxmp@embed@packet@pdftex
  \else
    \def\ccxmp@driver{hdvipdfm}%
    \ifx\ccxmp@driver\Hy@driver
      \ccxmp@embed@packet@dvipdfm
    \else
      \def\ccxmp@driver{hxetex}%
      \ifx\ccxmp@driver\Hy@driver
        \ccxmp@embed@packet@xetex
      \else
        \@ifundefined{pdfmark}{}{%
          \ccxmp@embed@packet@pdfmark
        }%
      \fi
    \fi
  \fi
  \endgroup
}
%% 
%% Copyright (C) 2014 Digital Curation Centre, University of Edinburgh
%% <info@dcc.ac.uk>
%%
%% End of file `dccpaper-base.tex'.
